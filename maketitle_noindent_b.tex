\maketitle
\noindent
{\bf 
Strong spin-orbit coupling (SOC) leads to numerous novel electronic phases, such as topological insulators and spin-orbit Mott insulators, but tuning the strength of SOC across quantum critical points (QCP) of such phases remains a challenge. Here, we show that competing energy scales of SOC, magnetic exchange coupling, and crystal electric fields in 4$d$ transition-metal oxides naturally poise them near a QCP. Our combined inelastic neutron scattering and theoretical investigation demonstrates in the archetypal 4$d$ Mott insulator \CRO that a crystal electric field effectively tunes the balance between a quantum paramagnet and an antiferromagnetic phase, favored by SOC and magnetic exchange coupling, respectively. Virtual magnetic dipoles emerge from quantum fluctuation between spin-orbit split levels, which upon condensation give rise to a ``Higgs'' amplitude mode most prominent near the QCP. Moving further into the antiferromagnetic phase, the distinct spin-orbital-lattice dynamics in \CRO is captured by a pseudospin Hamiltonian formally describing a spin nematic, thus providing a new framework to interpret SOC driven phenomena in 4$d$ transition-metal oxides.
}