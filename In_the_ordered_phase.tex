In the ordered phase, the pseudospin description holds for $\delta\gtrsim0.6$---almost everywhere except very near the QCP. The magnetic Hamiltonian is derived in terms of pseudospins ($\tilde{\mathbf{S}}$) as
\begin{align*}
H &= \sum_{\langle ij\rangle } 
[J_{xy} (\tilde{S_{ix}} \tilde{S_{jx}}+\tilde{S_{iy}} \tilde{S_{jy}}) + J_z \tilde{S_{iz}} \tilde{S_{jz}} \\
&\pm A (\tilde{S_{ix}} \tilde{S_{jy}}+\tilde{S_{iy}} \tilde{S_{jx}}) + \lambda_{\mathrm{eff}} \tilde{S_z}^2 +\Delta' \tilde{S_y}^2],
\end{align*}
\noindent
where $J_{xy/z}$ are the nearest-neighbor exchange, $A$ is the bond-directional exchange that naturally arise with SOC, and \leff and $\Delta'$ represent single-ion anisotropies in the orthorhombic local symmetry. We neglect the terms arising from the tilting and rotation of RuO$_6$ octahedra, such as Dzyaloshinskii-Moriya interactions, which is important for understanding the static moment structure but is not a major factor affecting the spin-wave spectra.  Although the form of the Hamiltonian could have been deduced from symmetry arguments, our derivation %provides the microscopic origin of the Hamiltonian, and 
links the coupling constants to the microscopic parameters: $J_{xy/z}$ and $A$ are functions of $\delta$ and $\eta$ (Supplementary Fig.~2), and \leff is a function of $\lambda$ and $\delta$ (Fig.~2). These relations provide the microscopic origin of the coupling constants (listed in Fig.~4 caption), in particular that strong SOC leads to the large energy scale of \leff far exceeding $J_{xy/z}$, which effectively renders an XY-like magnetism.
  
  
  