\noindent
SOC competes with various other effects in transition-metal oxides that tend to suppress it; for example, crystal electric fields, charge hopping, and superexchange interactions intermix orbital states with different quantum numbers and quench orbital angular momentum. In 5$d$ TMO, SOC dominates over these effects and fosters exotic spin-orbit driven phases. Examples range from correlated analogs of topological insulators to unconventional magnetism  and superconductivity. However, the low-energy physics is confined to the lowest spin-orbit multiplets in the strong SOC limit. On the other hand, spin-orbit effects are rather fragile in 3$d$ TMO and often are suppressed to an insignificant level. Rare examples of non-Heisenberg magnets are found in cobaltates with near-ideal cubic symmetry. As we shall see in this Article, 4$d$ TMO are poised in a `sweet spot' for unconventional electronic properties that emerge from competition of SOC and other traditional players of correlated electron physics. 