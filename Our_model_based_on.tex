Our model, based on the strong-coupling limit, is well justified for \CROns which is established as a Mott insulator, and requires no fine-tuning of the parameters to fit the data. Further, our model provides a microscopic basis for the coupling constants that describe the experimental data, and explains the origin of the Hamilonian that empirically describes the spin-nematic phase: the `spins' represent the highly spin-orbit entangled electronic state that results from the close competition between relativistic SOC and crystal electric fields. Although such spin-orbit entangled states are commonplace in 5$d$ TMOs, the comparable energy scales of $J$ and $\lambda$ in 4$d$ TMOs allows intermixing of multiplets with different $J_{\mathrm {eff}}$ while still preserving the strong spin-orbit entanglement, which is a key distinction from the 5$d$ TMO physics that allows for the unique QCP in 4$d$ TMOs. Further, the fact that substitution of Ca by Sr in \CRO results in a $\delta$$\approx$0 suggests that the QCP may be within a realistic range, if not pre-empted by a Mott transition.