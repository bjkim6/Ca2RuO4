Our two-parameter model provides an unbiased description of the data, and makes a definite prediction of the amplitude mode. Although \CRO is not in the immediate vicinity of the QCP, the condensate density is far from saturation, and the amplitude mode still retains a sizable intensity to be experimentally observed (Fig.~3h and 3i). Figure 4 shows the best fit of the above model (Fig.~1b) to the experimental data at three representative $\mathbf{q}$ points. Here, we allow the coupling constants to vary independently of one another around the values obtained from the microscopic parameters, considering that our model is minimal, and that the coupling constants in real materials are always renormalized by various effects not taken into account. This results in a better fitting of the transverse mode dispersion. An unambiguous identification of the amplitude mode is hindered by an intense optical phonon mode that overlaps with the amplitude mode in much of the momentum space. This phonon mode can be isolated by measuring in a high-$\mathbf{q}$ Brillouin zone where the magnetic form factor of Ru(IV) suppresses the magnetic signal below the detection level. The spectra are fitted with together with the measured phonon lineshape after a correction for its ${\mathbf q}$ dependence. This procedure reveals  a feature that very well fits the description of the amplitude mode; for instance, in the spectrum for $\mathbf{q}$=($\pi,\pi$), there is a clear peak around $\approx$35 meV, which is not accounted for by either the transverse mode or the phonon mode. A future polarized neutron scattering experiment can provide conclusive evidence for the amplitude mode origin of this feature.