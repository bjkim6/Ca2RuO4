Figure 1 plots the INS spectra along high-symmetry directions comprising spin-wave dispersion in the energy range 14 $\lesssim$ $\hbar\omega$ $\lesssim$ 45 meV and a weakly dispersing optical phonon mode around 42 meV. Their identification as magnetic and phonon modes follows from their intensity dependence on temperature and the magnitude of the momentum transfer ($\mathbf{q}$) (See Supplementary Fig.~1). The spin-wave emanates from the ordering vector ($\pi$,$\pi$) with a gap of $\approx$14 meV, and reaches a maximum at the $\Gamma$ point, qualitatively deviating from the well-known Heisenberg dispersion relations (Fig.~1c). Rather, the dispersion more resembles that of a XY magnet, which has a maximum at the $\Gamma$ point. This extreme magnetic anisotropy axiomatically implies strong SOC, which is consistent with the general expectation based on the atomic relativistic SOC constant $\zeta$=161 meV for Ru(IV) and indicated in earlier experiments, but not included in most theories. In such theories, it is generally agreed upon that the Mott insulation occurs in the half-filled $yz$ and $zx$ orbitals (meaning fully occupied $xy$ orbital), which necessarily implies a complete quenching of the orbital degrees of freedom, and hence that \CRO is a spin-one Heisenberg antiferromaget.