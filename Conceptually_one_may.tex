Conceptually, one may find a useful analogy to the magnetic field driven QCPs in dimer systems, such as TlCuCl$_3$. In these systems, a similar singlet-triplet structure is realized from a pairs of spins on each dimer, and the description of its QCP follows a similar formalism as treated here. In fact, the amplitude modes in \CRO and the dimer systems are both condensed-matter realizations of the Higgs mode, an elementary particle in the Standard Model, exemplifying universality of the fundamental laws of nature replicated across different settings over different energy scales. \CRO extends the few Kelvin, meV scale physics of dimer systems to a room-temperature scale, which means robust quantum effects that may be exploited for practical applications. Although such a large energy scale QCP cannot be tuned by laboratory-scale magnetic fields, ruthenates offer an alternative means of tuning via the crystal electric fields. Moreover, the singlet-triplet structure realized within a single ion removes the stringent requirement of dimer structure in the crystal lattice and thus has a wider applicability for transition-metal compounds containing 4$d$ ions with the low-spin $d^4$ configuration.