
\vspace{20 pt}
\noindent
{\bf Methods}

\vspace{10 pt}
\noindent
{\bf Sample synthesis \& characterization}

\noindent
%
Single crystals of \CRO were grown by the floating zone method with RuO$_{2}$
self-flux~\cite{Nakatsuji97,Cao97}. The crystals had lattice parameter
$a$=5.415~\r{A}, $b$=5.512~\r{A}, and $c$=11.944~\r{A}, as determined by x-ray
powder diffraction, in good agreement with the parameters reported in
literature\cite{Braden_1998} for the ``S'' phase with short $c$-axis lattice parameter.
The magnetic ordering temperature T=110~K was determined using magnetization
measurements in a Quantum Design SQUID-VSM device.

\vspace{10 pt}
\noindent
{\bf Inelastic neutron scattering}

\noindent
%
For the INS measurements, we co-aligned about 100 pieces of single crystals
with a total mass of $\sim$1.5~g into a mosaic on Al plates. The in-plane and
$c$-axis mosaicity of the aligned aligned crystal assembly were
$\lesssim$3.2$^\circ$ and  $\lesssim$2.7$^\circ$, respectively. The INS
measurements were performed on the ARCS time-of-flight chopper spectrometer at
the Spallation Neutron Source, Oak Ridge National Laboratory, Tennessee, USA.
The incident neutron energy was 100~meV. The Fermi chopper and  $T_{0}$ chopper
frequencies were set to 600 and 90 Hz, respectively, to optimize the neutron
flux and energy-resolution. The measurements were carried out at 5 K. The
sample was mounted with $(HOL)$ plane horizontal. The crystal was rotated over
90$^\circ$  about the vertical $\mathbf{c}$-axis with a step size of 1$^\circ$.
At each step data were recorded over a deposited proton charge of $8.33
e^{11}$nAh ($3 e^{12}$p$C$ proton charge, $\sim 36$~minutes)
and then converted into 4D $S(Q,\omega)$ using the Horace software\cite{horace}
and normalized with a recent Vanadium calibration.

\vspace{10 pt}
\noindent
{\bf Polarized inelastic neutron scattering}

\noindent
%
For the polarized triple axis measurement we remounted the crystals from the
time-of-flight experiment on Si plates and augmented it with further crystals
to obtain a total sample mass of $\sim$3~g.
%
The mosaicity of this sample was $\lesssim$3.2$^\circ$ and 
$\lesssim$2.6$^\circ$ for in-plane and $c$-axis, respectively.
%
The experiment was performed on the IN20 three-axis-spectrometer installed on
the H13 thermal beam tube at the Institut Laue-Langevin, Grenoble, France. For
the $xyz$~polarization analysis, we used a Heusler (111) monochromator as well
as a Heusler (111) analysator in combination with the Helmholtz coil at the
sample position.
%
Throughout the experiment we used a fixed $k_F$=2.663\r{A}$^{-1}$ and performed
polarization analysis along energy and H scans at ($\pi$,$\pi$) and (0,0),
keeping L as small as possible without violating the kinematic restrictions.



\vspace{10 pt}
\noindent
{\bf Theory}

\noindent
%
With the Helmholtz coil at the sample position, the neutron spin can be
oriented in arbitrary positions. It is useful to describe the scattering
process in a reference frame, such that $x\parallel q$, $y\perp q$ is in the
scattering plane and $z$ is vertical.
%
In the polarization analysis different scattering channels with the spin
pointing along $x$, $y$, and $z$ are investigated. Depending if the spin was flipped
during the scattering process, these are further separated
into spin-flip and non-spin-flip channels.

\noindent
The magnetic double differential cross section for each channel is given by
\cite{Schaerpf_1993}
%
\begin{equation}
  \left(\frac{\partial^2\sigma}{\partial \Omega \partial E'}\right)_{ss'}^{i}
  = \left( \frac{m_n}{2\pi\hbar^2}\right)^2 \frac{k'}{k} (\gamma \mu_n)^2
    \Gamma_{ss'}^{i}(\vec{r},\omega)
\end{equation}
%
Here, $s$ and $s'$
denote if the spin was flipped during the scattering process. The initial and
final moment of the neutron is given by $k$ and $k'$, respectively, and $\gamma
\mu_n$ is the product of the gyromagnetic moment of the neutron with its
magnetic moment. The symbol $\Gamma_{ss'}^{i}(\vec{r},\omega)$ stands for all
matrix elements $M_{ii}(\vec{r},\omega)$, that contribute for the respective
channel. We here reproduce an incomplete listing of these functions, that
captures all the effects needed to describe our experiment. For a more complete
overview see \cite{Schaerpf_1993}
%
\begin{equation}
  \begin{pmatrix}
    \Gamma_{sf}^x (\vec{r},\omega)\\
    \Gamma_{sf}^y (\vec{r},\omega)\\
    \Gamma_{sf}^ z (\vec{r},\omega)
  \end{pmatrix}
  =
  \begin{pmatrix}
    0 & 1 & 1 \\
    0 & 0 & 1 \\ 
    0 & 1 & 0 
  \end{pmatrix}
  \begin{pmatrix}
    M_{xx}(\vec{r},\omega) \\
    M_{yy}(\vec{r},\omega) \\
    M_{zz}(\vec{r},\omega)
  \end{pmatrix}
\end{equation}
%
Combining these equations, the scattering to the pure magnetic moments in $y$
and $z$ direction can be calculated while at the same time the background
signal is completely eliminated:
%
\begin{equation}
  \begin{pmatrix} 
    M_{\perp}(\vec{r},\omega) \\
    M_{yy}(\vec{r},\omega) \\ 
    M_{zz}(\vec{r},\omega) 
  \end{pmatrix} 
  =
  \begin{pmatrix} 
    2 & -1 & -1 \\ 
    1 & 0 & -1 \\ 
    1 & -1 & 0 
  \end{pmatrix}
  \begin{pmatrix} 
    \Gamma_{sf}^x (\vec{r},\omega)  \\
    \Gamma_{sf}^y(\vec{r},\omega) \\
    \Gamma_{sf}^z (\vec{r},\omega)
  \end{pmatrix}
\end{equation}
%
A further transformation is needed to
transform these quantities into the frame of the sample cyrstal. In case of an
(H0L) scattering geometry, the following holds true:
%
\begin{equation} 
  \begin{pmatrix} M_{\perp} \\ 
    M_{yy} \\ 
    M_{zz}
  \end{pmatrix}_{(H0L)} 
  = 
  \begin{pmatrix} 
    \sin(\alpha)  & 1 & \cos(\alpha) \\
    \sin(\alpha)  & 0 & \cos(\alpha) \\ 
    0 & 1 & 0 
  \end{pmatrix} 
  \begin{pmatrix}
    M_a \\ M_b \\ M_c 
  \end{pmatrix} 
\end{equation}
