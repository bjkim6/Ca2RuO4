\vspace{ 10 pt}
\noindent
{\bf Spin-wave spectra reveals XY-type magnetism}

\noindent
Figures 1a and 1b plot, respectively, constant-energy maps and INS spectra along high-symmetry directions, comprising spin-wave dispersion in the energy range 14 $\lesssim$ $\hbar\omega$ $\lesssim$ 45 meV and phonon modes above $\sim$50 meV. The magnetic nature of the former is explicitly confirmed by using polarized neutrons (see Fig.~X), and the non-magnetic nature of the latter is inferred from exhaustion of all magnetic modes and also through comparison with phonon modes in Raman scattering data (See Supplementary Fig.~1). The spin-wave emanates from the ordering vector ($\pi$,$\pi$) with a gap of $\approx$14 meV, and reaches a maximum at the $\Gamma$ point, qualitatively deviating from the well-known Heisenberg dispersion relations (Fig.~1c). Rather, the dispersion more resembles that of a XY magnet, which has a maximum at the $\Gamma$ point. This extreme magnetic anisotropy axiomatically implies strong SOC, which is consistent with the general expectation based on the atomic relativistic SOC constant $\zeta$=161 meV for Ru(IV) and indicated in earlier experiments, but not included in most theories. This negligence of SOC has led to the view that the Mott insulation occurs in the half-filled $yz$ and $zx$ orbitals (meaning fully occupied $xy$ orbital), which necessarily implies a complete quenching of the orbital degrees of freedom, and hence that \CRO is a spin-one Heisenberg antiferromaget. This long-held view becomes untenable with our INS data.

  