Why has strong SOC in \CRO eluded detection for more than two decades? The fact that INS data has been unavailable until now partly accounts for it, but, in retrospect, it should have been noticed from the large canting angle of the magnetic moment, evident from the net ferromagnetic moment of ($\sim$ 0.3 $\mu_{\mathrm{B}}$/Ru) in the metamagnetic state above a moderate magnetic field of $\approx7$ T. The magnetic structure is very similar to that of Sr$_2$IrO$_4$, in which strong SOC rigidly fixes the magnetic moment with respect to the IrO$_6$ octahedra, and canted moments result from the octahedra rotation. However, as we have seen, the manifestion of SOC is \CRO is quite distinct from that in any of the known 5$d$ TMO; for example, Heisenberg antiferromagnet Sr$_2$IrO$_4$,  Ising-like Sr$_3$Ir$_2$O$_7$, or Na$_2$IrO$_3$ with dominant bond-directional magnetic coupling. It is also different from 3$d$ TMO such as cobaltates or ferrites, where SOC results in exotic non-collinear magnetic structures.